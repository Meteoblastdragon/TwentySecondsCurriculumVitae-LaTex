%%%%%%%%%%%%%%%%%%%%%%%%%%%%%%%%%%%%%%%%%
% Curriculum Vitae Timeline/CV
% LaTeX Template
% Version 1.0 (5/23/19)
%
% Original author:
% Carmine Spagnuolo (cspagnuolo@unisa.it) with major modifications by
% Vel (vel@LaTeXTemplates.com) and finally by
% Augusto Cunha (contact@augustoicaro.com)
%
% License:
% The MIT License (see included LICENSE file)
%
%%%%%%%%%%%%%%%%%%%%%%%%%%%%%%%%%%%%%%%%%

%----------------------------------------------------------------------------------------
%	PACKAGES AND OTHER DOCUMENT CONFIGURATIONS
%----------------------------------------------------------------------------------------

\documentclass[a4paper]{twentysecondcv} % a4paper or latterpaper
\usepackage{anyfontsize}
%----------------------------------------------------------------------------------------
%	 PERSONAL INFORMATION
%----------------------------------------------------------------------------------------

% If you don't need one or more of the below, just remove the content leaving the command, e.g. \cvnumberphone{}

\profilepic{alice.jpeg} % Profile picture

\cvname{Alice} % Your name
\cvjobtitle{Aventureira} % Job title/career

\cvdate{} % Date of birth
\cvaddress{Brasil} % Short address/location, use \newline if more than 1 line is required
\cvnumberphone{+55 82 99999-9999} % Phone number
\cvsite{http://en.wikipedia.org} % Personal website
\cvmail{alice@wonderland.com} % Email address
\cvgithub{alicewonderland} % Github user account
\cvlinkedin{alicewonderland} % Linkedin user account

%----------------------------------------------------------------------------------------

\begin{document}

%----------------------------------------------------------------------------------------
%	 Profile
%----------------------------------------------------------------------------------------

\profile{Perfil}{Mussum Ipsum, cacilds vidis litro abertis. Todo mundo vê os porris que eu tomo, mas ninguém vê os tombis que eu levo! Quem num gosta di mim que vai caçá sua turmis! Mé faiz elementum girarzis, nisi eros vermeio. Paisis, filhis, espiritis santis.} % To have no About Me section, just remove all the text and leave \aboutme{}

%----------------------------------------------------------------------------------------
%	 ABOUT ME
%----------------------------------------------------------------------------------------

%\aboutme{About Me}{Suco de cevadiss, é um leite divinis, qui tem lupuliz, matis, aguis e fermentis. Praesent malesuada urna nisi, quis volutpat erat hendrerit non. Nam vulputate dapibus. Copo furadis é disculpa de bebadis, arcu quam euismod magna. Aenean aliquam molestie leo, vitae iaculis nisl.} % To have no About Me section, just remove all the text and leave \aboutme{}

%----------------------------------------------------------------------------------------
%	 Interests
%----------------------------------------------------------------------------------------

\interest{Outros Interesses}{A heroína e senhoradora da terra das maravilhas, Alice é o personagem principal.}

%----------------------------------------------------------------------------------------
%	 SKILLS
%----------------------------------------------------------------------------------------

% Skill bar section, each skill must have a value between 0 an 5 (float)
\skills{Habilidades}{{Educada/5},{Persuadir Coelhos/4.8}}

%------------------------------------------------

% Skill text section, each skill must have a value between 0 an 6
%\skillstext{}

%----------------------------------------------------------------------------------------

\makeprofile % Print the sidebar

%----------------------------------------------------------------------------------------
%	 EXPERIENCE
%----------------------------------------------------------------------------------------

\section{Experiência}

\begin{twenty} % Environment for a list with descriptions
    \twentyitemtime{mainblue}{white}{Jan/2014}{Momento}{Cargo}{Empregador, cidade, estado}{\makeList{Tente usar o método S.M.A.R.T. para descrever o seu papel.;Mais vale um bebadis conhecidiss, que um alcoolatra anonimis.;Suco de cevadiss deixa as pessoas mais interessantis;Per aumento de cachacis, eu reclamis.;Interessantiss quisso pudia ce receita de bolis, mais bolis eu num gostis.}}
    \twentyitemtime{mainblue}{white}{Jan/2009}{Jan/2014}{Cargo}{Empregador, cidade, estado}{\makeList{Mais vale um bebadis conhecidiss, que um alcoolatra anonimis.;Suco de cevadiss deixa as pessoas mais interessantis;Per aumento de cachacis, eu reclamis.;Interessantiss quisso pudia ce receita de bolis, mais bolis eu num gostis.}}


\end{twenty}

%----------------------------------------------------------------------------------------
%	 EDUCATION
%----------------------------------------------------------------------------------------

\section{Formação Acadêmica}

\begin{twenty} % Environment for a list with descriptions
	\twentyitemtime{mainblue}{white}{Aug/2018}{Present}{Ph.D. Field}{University}{Mussum Ipsum, cacilds vidis litro abertis. Quem num gosta di mé, boa gentis num é. Todo mundo vê os porris que eu tomo, mas ninguém vê os tombis que eu levo! Pra lá , depois divoltis porris, paradis. Nec orci ornare consequat. Praesent lacinia ultrices consectetur. Sed non ipsum felis.}
	\twentyitemtime{mainblue}{mainblue}{Aug/2016}{Jul/2018}{M.Sc. Fielde}{University}{Delegadis gente finis, bibendum egestas augue arcu ut est. Atirei o pau no gatis, per gatis num morreus. Per aumento de cachacis, eu reclamis. Vehicula non. Ut sed ex eros. Vivamus sit amet nibh non tellus tristique interdum.}
	\twentyitemtime{mainblue}{mainblue}{Aug/2010}{Jul/2010}{B.Sc. Field}{University}{Quem num gosta di mim que vai caçá sua turmis! Si u mundo tá muito paradis? Toma um mé que o mundo vai girarzis! Viva Forevis aptent taciti sociosqu ad litora torquent. Interessantiss quisso pudia ce receita de bolis, mais bolis eu num gostis.}
	%\twentyitem{<dates>}{<title>}{<location>}{<description>}
\end{twenty}
\begin{twenty}
\twentyitem{2007-2010}{B.Sc. Field}{Univesity}{Mussum Ipsum, cacilds vidis litro abertis. Quem num gosta di mé, boa gentis num é. Todo mundo vê os porris que eu tomo, mas ninguém vê os tombis que eu levo! Pra lá , depois divoltis porris, paradis. Nec orci ornare consequat. Praesent lacinia ultrices consectetur. Sed non ipsum felis.}
\end{twenty}

%----------------------------------------------------------------------------------------
%	 PUBLICATIONS
%----------------------------------------------------------------------------------------

%\section{Publications}

%\begin{twentyshort} % Environment for a short list with no descriptions
%	\twentyitemshort{1865}{Chapter One, Down the Rabbit Hole.}
%	\twentyitemshort{1865}{Chapter Two, The Pool of Tears.}
%	\twentyitemshort{1865}{Chapter Three,  The Caucus Race and a Long Tale.}
%	\twentyitemshort{1865}{Chapter Four,  The Rabbit Sends a Little Bill.}
%	\twentyitemshort{1865}{Chapter Five,  Advice from a Caterpillar.}
	%\twentyitemshort{<dates>}{<title/description>}
%\end{twentyshort}

%----------------------------------------------------------------------------------------
%	 AWARDS
%----------------------------------------------------------------------------------------

\begin{twenty} % Environment for a short list with no descriptions
	\twentyitem{1998}{Melhor livro de fantasia de todos os tempos antes de 1990.}{}{Alguma descrição.}
	\twentyitem{1987}{Melhor livro de fantasia de todos os tempos.}{}{Alguma descrição.}
	%\twentyitemshort{<dates>}{<title/description>}
\end{twenty}

%----------------------------------------------------------------------------------------
%	 OTHER INFORMATION
%----------------------------------------------------------------------------------------

\section{Outras Informações}

Mussum Ipsum, cacilds vidis litro abertis. Praesent vel viverra nisi. Mauris aliquet nunc non turpis scelerisque, eget. Casamentiss faiz malandris se pirulitá. Detraxit consequat et quo num tendi nada. Praesent malesuada urna nisi, quis volutpat erat hendrerit non. Nam vulputate dapibus.

Mauris nec dolor in eros commodo tempor. Aenean aliquam molestie leo, vitae iaculis nisl. Em pé sem cair, deitado sem dormir, sentado sem cochilar e fazendo pose. Sapien in monti palavris qui num significa nadis i pareci latim. Si u mundo tá muito paradis? Toma um mé que o mundo vai girarzis!

Todo mundo vê os porris que eu tomo, mas ninguém vê os tombis que eu levo! Suco de cevadiss, é um leite divinis, qui tem lupuliz, matis, aguis e fermentis. Paisis, filhis, espiritis santis. Mé faiz elementum girarzis, nisi eros vermeio.

%----------------------------------------------------------------------------------------
%	 INTERESTS
%----------------------------------------------------------------------------------------

%\section{Qualifications}

%\begin{itemize}
%    \item Proficiency in Linux systems: CentOS, Debian, Ubuntu, Elementary OS.
%    \item Experience in Bash, Python, C++, R, Vim, Subversion, Git, OpenGL.
%    \item Experience using Docker and Singularity.
%    \item Experience in scrum, Jira, Jenkins.
%    \item Experience in remote access, network management, system administration.
%    \item Experience in math, statics, algorithms, solving problems, supervised machine learning problems.

%\end{itemize}

%----------------------------------------------------------------------------------------
%	 SECOND PAGE EXAMPLE
%----------------------------------------------------------------------------------------

%\newpage % Start a new page

%\makeprofile % Print the sidebar

%\section{Other information}

%\subsection{Review}

%Alice approaches Wonderland as an anthropologist, but maintains a strong sense of noblesse oblige that comes with her class status. She has confidence in her social position, education, and the Victorian virtue of good manners. Alice has a feeling of entitlement, particularly when comparing herself to Mabel, whom she declares has a ``poky little house," and no toys. Additionally, she flaunts her limited information base with anyone who will listen and becomes increasingly obsessed with the importance of good manners as she deals with the rude creatures of Wonderland. Alice maintains a superior attitude and behaves with solicitous indulgence toward those she believes are less privileged.

%\section{Other information}

%\subsection{Review}

%Alice approaches Wonderland as an anthropologist, but maintains a strong sense of noblesse oblige that comes with her class status. She has confidence in her social position, education, and the Victorian virtue of good manners. Alice has a feeling of entitlement, particularly when comparing herself to Mabel, whom she declares has a ``poky little house," and no toys. Additionally, she flaunts her limited information base with anyone who will listen and becomes increasingly obsessed with the importance of good manners as she deals with the rude creatures of Wonderland. Alice maintains a superior attitude and behaves with solicitous indulgence toward those she believes are less privileged.

%----------------------------------------------------------------------------------------

\end{document}